\pagenumbering{roman}

\begin{abstract} 

\noindent 
notater: 
"I think the next centry will be the centry of complexity" 
- Stephen Hawking 

We are surrounded by systems that are hoplessly complicated. Consider for example the society that requires cooperation between billions of individuals, or communications infrastrucures that integrate billions of cell phones with computers and satellites. Our ability to reason and comprehend our world requires the coherent acivity of billions of neurons in our brain. 

These systems are collectively called complex systems, capuring the fact that it is difficult to derive their collective behavoir from knodledge of the system's components. Given the important role complex systems play in our daily life, in science and in economy, their understanding, mathematical description, prediction, and eventually control is one of teh major intellectual and scientific challenges of the 21st centry. 

A key discovery of network science is that the architecture of networks emerging in various domains of science, nature, and technology are similar to each other, a conseuence of being governed y the same organization principles. Cosequently we can use a common set of mathematical tools to explore these systems. 

In summary, while many disciplines have made the important controbutions to network science, the emergence of a new field was parly made possible by data avaailabilty, offering accurate maps of networks encountered in different disciplines. These diverse maps allowed network scientists to identify the universal properties of various network characteristics. 


\end{abstract}

\renewcommand{\abstractname}{Acknowledgements}
\begin{abstract}


%Pål for friheten til å velge master oppgave selv 
%mamma, pappa og David for å alltid støtte meg 
	\vspace{1cm}
	\hspace*{\fill}\texttt{Sandra Lekve}\\ 
	\hspace*{\fill}\today
\end{abstract}
\setcounter{page}{1}
\newpage