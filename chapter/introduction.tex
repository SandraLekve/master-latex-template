\chapter{Introduction}


Management: Uncovering the Internal Structure of an Organization

While management tends to rely on the official chain of commands, it is increasingly evident that the informal network, capturing who really communicates with whom, play the most important role in the success of an organization. Accurate maps of such organizational networks can expose the potential lack of interactions between key units, help identify individuals who play an important role in bringing different departments and products products together, and help higher management diagnose diverse organizational issues. Furthermore, there is increasing evidence in the management literature that the productivity of an employee is determined by their position in this informal organizational network. 

Therefore, numerous companies, like Maven 7, Activate Networks or Orgent, offers tools and methodologies to map out the true structure of an organization. These companies offer a host of services, from identifying opinion leaders to reducing employee churn,  optimizing knowledge and product diffusion and designing teams with the diversit, size and expertise to be the mose effective for specific tasks. Established firms, from IBM to SAP, have added social networking capabilities to their business. Overall, network science tools are indispensable in management and business, anhancing productivity and boosting innovation within an organization. 

\section{Background}

Lorem ipsum dolor sit amet, cu graecis propriae sea. Eam feugiat docendi an, ei scripta blandit pri. Nonumes delicata reprimique nam ut. Eu suas alterum concludaturque est, ferri mucius sensibus id sed~\cite{raftAlg}.

We can do glossary for acronymes and abriviations also: \gls{saas}. As you see the first time it is used, the full version is used, but the second time we use \gls{saas} the short form is used. It is also a link to the lookup.


\subsection{Listings}
You can do listings, like in Listing~\ref{ListingReference}
\begin{lstlisting}[caption={[Short caption]Look at this cool listing. Find the rest in Appendix~\ref{Listing}},label=ListingReference]
$ java -jar myAwesomeCode.jar
\end{lstlisting}

You can also do language highlighting for instance with Golang:
And in line~\ref{LineThatDoesSomething} of Listing~\ref{ListingGolang} you can see that we can ref to lines in listings.

\begin{lstlisting}[caption={Hello world in Golang},label=ListingGolang,escapechar=|]
package main

import "fmt"

func main() {
    fmt.Println("hello world") |\label{LineThatDoesSomething}|
}

\end{lstlisting}

\subsection{Figures}

Example of a centred figure
\begin{figure}[H]
    \centering
    \includegraphics[scale=0.5]{figures/Flowchart}
    \caption{Caption for flowchart}
  	\medskip 
	\hspace*{15pt}\hbox{\scriptsize Credit: Acme company makes everything \url{https://acme.com/}}
    \label{FlowchartFigure}
\end{figure}

\subsection{Tables}

We can also do tables. Protip: use \url{https://www.tablesgenerator.com/} for generating tables.
\begin{table}[H]
\centering
\caption{Caption of table}
\label{TableLabel}
\begin{tabular}{|l|l|l|}
\hline
Title1 & Title2 & Title3 \\ \hline
data1  & data2  & data3  \\ \hline
\end{tabular}
\end{table}

\subsection{\gls{git}}

\gls{git} is fun, use it!